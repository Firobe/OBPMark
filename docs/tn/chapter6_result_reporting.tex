\chapter{Test Result Reporting}
This section contains information regarding the information to be provided when reporting benchmark results.
\\
\\
Test results shall be submitted to: obpmark@esa.int and david.steenari@esa.int for verification.
\\
\\
Verified results can be re-published on the OBPMark website. Proprietary or company confidential information which could be required for the verification of the results, can be removed prior to publication if requested by the submitting company or entity.

\section{General Implementation Description}
The following shall be included as part of the result reporting: 

A general description of the test-setup and the DUT (Device Under Test) or DUTs, including list of auxilirary equipment used.

A short description of the PCBs where the DUT is implemented. 

\subsection{Parallelisation and Optimizations}
The used parallelisation scheme shall be reported.

Processing kernel functions optimizations implemented (None, intrinsic functions for DSP instructions, and/or assembly code optimization) shall be reported.

\subsection{Test Parameters}

Number of times the test has been carried out. 

\section{Implementation Metrics}

\subsection{Implementation Metrics for FPGA}

FPGA utilization metrics:
\begin{itemize}
    \item Number of used logic resources (e.g. LUTs)
    \item Number of used dedicated DSP blocks
    \item Number of used memory blocks
\end{itemize}

All FPGA implementation utilization metrics shall be provided both as absolute number of elements, and as a precentage of the device total.

\begin{itemize}
    \item Other used 
\end{itemize}

\section{Power Metrics}

\begin{itemize}
    \item Description of power measurement setup (e.g. ampere metre, internal device reporting, etc.) 
\end{itemize}

\newpage
\section{Reporting Templates}
For each of the benchmarks there are specific parameters to measure and report. In the following subsections, respective reporting templates are provided for each of the benchmarks. 

The following reporting template is common for all benchmarks, and should always be included. 

\begin{table}[!h]
    \centering
    \begin{tabular}{|l|l|l|}
        \hline
        Parameter                           & Value  & Comment \\ \hline
        \hline
        Name of DUT(s)                      &  & \\ \hline
        DUT type (processor, FPGA, etc.)    &  & \\ \hline
        Number of devices/systems tested    &  & \\ \hline
    \end{tabular}
    \caption{Common Result Reporting Template}
    \label{tab:common_report_template}
\end{table}

The following template should be used whenever processing functions are implemented in FPGA. 

\begin{table}[!h]
    \centering
    \begin{tabular}{|l|l|l|}
        \hline
        Parameter                           & Value  & Comment \\ \hline
        \hline
        Total number of logic elements (e.g. LUTs) used     &  & \\ \hline
        Percentage of logic elements used of device total   &  & \\ \hline
        ... & & \\ \hline
        
    \end{tabular}
    \caption{Common FPGA Result Reporting Template}
    \label{tab:fpga_report_template}
\end{table}

\newpage
\subsection{Benchmark \#1.1 Result Reporting Template}

\begin{table}[!h]
    \centering
    \begin{tabular}{|l|l|l|}
        \hline
        Parameter               & Measured Valued   & Comment \\ \hline
        \hline
        Average \# of pixels/s  &  & \\ \hline
        Average Mbit/s    &  & \\ \hline
    \end{tabular}
    \caption{Benchmark \#1.1 Result Reporting Template}
    \label{tab:bm1_1_report_template}
\end{table}

\subsection{Benchmark \#1.2 Result Reporting Template}

\begin{table}[!h]
    \centering
    \begin{tabular}{|c|c|}
        \hline
         &  \\ \hline
         &  \\ \hline
    \end{tabular}
    \caption{Benchmark \#1.2 Result Reporting Template}
    \label{tab:bm1_2_report_template}
\end{table}

\newpage
\subsection{Benchmark \#2.1 Result Reporting Template}



\begin{table}[!h]
    \centering
    \begin{tabular}{|c|c|}
        \hline
        Total execution time & \\ \hline
        Prediction time & \\ \hline
        Encoding time & \\ \hline
    \end{tabular}
    \caption{Benchmark \#2.1 Result Reporting Template}
    \label{tab:bm2_1_report_template}
\end{table}

\subsection{Benchmark \#2.2 Result Reporting Template}

\begin{table}[!h]
    \centering
    \begin{tabular}{|c|c|}
        \hline
        Total execution time & \\ \hline
        DWT time & \\ \hline
        Encoding time & \\ \hline
    \end{tabular}
    \caption{Benchmark \#2.2 Result Reporting Template}
    \label{tab:bm2_2_report_template}
\end{table}

\subsection{Benchmark \#2.3 Result Reporting Template}

\begin{table}[!h]
    \centering
    \begin{tabular}{|c|c|}
        \hline
        Total execution time & \\ \hline
        Prediction time & \\ \hline
        Encoding time & \\ \hline
    \end{tabular}
    \caption{Benchmark \#2.3 Result Reporting Template}
    \label{tab:bm2_3_report_template}
\end{table}

\newpage
\subsection{Benchmark \#3.1 Result Reporting Template}

The results shall be measured in:

\begin{itemize}
    \item Throughput (Mbit/s)
    \item Throughput per power ( (Mbit/s) / W )
\end{itemize}


\begin{table}[!h]
    \centering
    \begin{tabular}{|c|c|}
        \hline
         &  \\ \hline
         &  \\ \hline
    \end{tabular}
    \caption{Benchmark \#3.1 Result Reporting Template}
    \label{tab:bm3_1_report_template}
\end{table}

\newpage
\subsection{Benchmark \#4.1 Result Reporting Template}

\begin{itemize}
    \item Execution time of FIR filter 
    \item Rate of FIR filter (samples/s)
\end{itemize}

\subsection{Benchmark \#4.2 Result Reporting Template}
TBA.

\subsection{Benchmark \#5.1 Result Reporting Template}
TBA.

\subsection{Benchmark \#5.2 Result Reporting Template}
TBA.